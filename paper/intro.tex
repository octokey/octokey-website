\section{Introduction}

Despite their many well-known flaws, passwords are still by far the most commonly used
authentication system on the web. However, as users sign up for ever more services, their use is
becoming increasingly unsustainable: demanding that users choose passwords with sufficient entropy,
use a different password for every site, not write them down and even rotate passwords periodically
is unrealistic for most internet services.

With phishing attacks and leaks of password databases unfortunately commonplace, we seek a better
solution. In this paper we focus on authentication in a consumer internet context, such as social
media or e-commerce websites and mobile apps. For a majority of such use cases, an authentication
system with the following properties is desirable:

\begin{itemize}
\item The system should minimize exposure to human error (such as phishing). Whilst human error
is unavoidable, especially amongst non-technical users, the system should be designed so as to
minimize the impact in the case of error.
\item It is not necessary, and often undesirable, for the system to authenticate the user as a
physical person. Many websites allow pseudonymous signup, which is desirable for reasons of privacy
or freedom of speech. The purpose of the authentication system is thus only to verify that the
current user is the same person as the one who originally signed up under a particular username; no
authentication is performed at signup.
\item The system should not depend on any trusted third party, to avoid risks arising from failure
of the third party (security compromise, going out of business, change of policies, etc).
\item The system should be easy for end-users to install and use, and easy for website owners to
deploy. Since user signup and user activity rates are important business metrics for many online
services, the system should make signup and login extremely easy and enjoyable for users.
\item Users should have the freedom to use the system on a variety of devices, including shared or
public computers that are only partially trusted, whilst minimizing their exposure to attackers.
For example, a user may choose log in to `unimportant' services (e.g. casual online games) on a
shared or public computer, knowing that their account on that service may be compromised; however,
they limit their use of `important' accounts (e.g. online banking) to trusted devices. Since there
is no generally agreed boundary between `important' and `unimportant' websites, the same
authentication system should be able to support both use cases.
\end{itemize}

In the next section we discuss various authentication systems that are used on the web, and explain
why none of them meet these requirements. We then introduce Octokey, an authentication protocol
which addresses these issues. We believe that Octokey could be a practicable solution to online
authentication in the near future, and we seek the information security community's feedback to
refine this proposal.

\section{Existing web authentication systems}
\subsection{Passwords}

Passwords have many of the desirable properties mentioned above: no dependence on any third party,
pseudonymity, simplicity of implementation, familiarity and cross-device compatibility. However,
remembering a large number of passwords is a big burden on users, and they are highly susceptible to
human error: use of weak passwords, reuse of the same password across different services, phishing
attacks and leaks of unhashed or weakly hashed password databases are sadly common.

Password manager products such as 1Password or LastPass make it feasible for users to maintain a
strong, unique password for each service they use. However, even when a unique password is used, an
attacker who succeeds in stealing a password (e.g. by phishing, keylogging, man-in-the-middle attack
or eavesdropping when it is accidentally sent over an unencrypted connection) has access to that
user account until the password is changed, which is often a long time.

Password managers also have grave exposure to human error. For example, consider a user who wishes
to log in to an `unimportant' website on an untrusted computer. It is very tempting for the user to
type the master passphrase of their password manager into the untrusted computer, in order to
decrypt the password database. This has the effect of exposing the user's entire password database
to an attacker.

\subsection{One-time passwords and two-factor auth}

One-time passwords (OTPs) offer a solution for users who want to log in to their accounts on
untrusted or partially trusted devices. The user must download a list of one-time passwords on a
trusted device, or register a cryptographic device that generates a pseudorandom sequence of OTPs,
or register a phone number to which OTPs can be sent on demand. Some services allow use of OTPs as
sole authentication mechanism, while others use them in conjunction with a regular password
(two-factor authentication).

The advantage of OTPs is that the exposure to attacks is limited to a single service, and ends when
the user clicks the logout button (assuming the session is correctly invalidated on the server, and
assuming that the attacker does not have a means of extending their privileges, e.g. by granting
themselves OAuth access to the account).

However, OTPs are so inconvenient to use that very few websites are willing to adopt them as their
only or primary authentication mechanism, and only the most security-conscious users are willing to
use them. Moreover, the common ``remember me on this device'' feature weakens two-factor
authentication (attackers just need to steal the ``remember me'' cookie as well as keylogging the
password).

\subsection{OpenID and OAuth}

OpenID~\cite{OpenID} is probably the best-known attempt to remove the need for a separate password
on every service. Some providers, such as Google Federated Login~\cite{GoogleOpenID}, combine OpenID
and OAuth~\cite{OAuth}.

OpenID does not solve the fundamental problem of authentication: it only delegates the problem to
the OpenID provider, who must then use some other authentication method (most commonly a password,
possibly in conjunction with a hardware token). This means that all the problems discussed above
apply to the OpenID provider, with an additional privilege escalation problem: if an attacker gains
access to the user's account with the OpenID provider, they can access any account associated with
that OpenID identity.

Moreover, in OpenID, the relying party needs to trust the OpenID provider to correctly authenticate
the user. If the OpenID provider is compromised, the relying party has no way of detecting
unauthorized logins from that OpenID provider. If the OpenID provider goes out of business, many
users of that provider will lose their ability to log in (unless they had set up delegation of their
identity URL, which is unrealistic to expect of non-technical users). Major online services rarely
act as a relying party in OpenID, because the risk of problems with external OpenID providers is too
great.

\subsection{Client certificates}

TLS~\cite{TLS} provides a mechanism for a client to authenticate itself to the server using a X.509
certificate. The server may specify the certificate authority from which it is willing to accept
certificates. By calculating a signature over the TLS key exchange messages, client certificate
authentication also provides protection against man-in-the-middle attacks (a signature for one TLS
connection cannot be reused on another TLS connection).

Client certificates are a good solution in situations where the physical identity of the user is
important: certificates might be issued by a government to its citizens (for filing taxes online),
or by a company to its employees (for accessing internal systems). For example, the government of
Estonia issues certificates on smart cards to its citizens through the national ID card scheme, and
the certificates are widely used for online banking authentication in Estonia.~\cite{Parsovs14}

Despite their advantages, client certificates are not widely used for consumer internet services.
Problems include:

\begin{itemize}
\item the server must decide which CAs to trust, or the server must itself act as a CA, in order to
avoid a trust dependency on a third party;
\item pseudonymous usage of services is difficult if the certificate is issued by an external CA;
\item the certificate (including identifying personal details such as the user's real name) is sent
unencrypted during the TLS handshake, making it visible to passive network attackers;
\item the user experience of installing and managing certificates is unfriendly in most web
browsers;
\item certificate validation can be difficult to set up on the server, as TLS termination is
typically performed on a separate server from the application server running business logic;
\item certificate validation can be computationally expensive, creating a risk of denial of
service;~\cite{Parsovs14}
\item revocation of certificates is easily missed, as it requires CRLs to be distributed or
OCSP~\cite{OCSP} to be used.
\end{itemize}

\subsection{BrowserID}
TODO
\subsection{FIDO}
TODO

Client certificate authentication can be performed by hardware tokens using the PKCS \#11
\emph{Cryptoki} API. This provides some protection against private keys being stolen by malware.
