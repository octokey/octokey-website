\section{Introduction}

For a majority of consumer web use cases, an authentication system with the following properties is
desirable:

\begin{itemize}
\item The system should not depend on any trusted third party, to avoid risks arising from failure
of the third party (security compromise, going out of business, change of policies etc).
\item It is not necessary, and often undesirable, for the system to authenticate the user as a
physical person. Many websites allow pseudonymous signup, which is desirable for privacy, freedom of
speech and other reasons. The purpose of the authentication system is thus only to verify that the
current user is the same person as the one who originally signed up under a particular username; no
authentication is performed at signup.
\item To ease adoption, the system should be easy for end-users to install and use, and easy for
website owners to deploy. Since user signup and user activity rates are important business metrics
for many websites, the system should make signup and login extremely easy and enjoyable for users.
\item The system should minimize its exposure to human error. Whilst human error is unavoidable,
especially amongst non-technical users, the system should be designed so as to minimize the impact
in the case of error.
\item Users should have the freedom to use the system on a variety of devices, including shared or
public computers that are only partially trusted, whilst minimizing their exposure to attackers.
For example, a user may choose log in to `unimportant' services (e.g. casual online games) on a
shared or public computer, knowing that their account on that service may be compromised; however,
they limit their use of `important' accounts (e.g. online banking) to trusted devices. Since there
is no generally agreed boundary between `important' and `unimportant' websites, the same
authentication system should be able to support both use cases.
\end{itemize}


\section{Existing web authentication systems}
\subsection{Passwords}

Despite their many well-known flaws, passwords are still by far the most commonly used
authentication system on the web. They have many of the desirable properties mentioned above: no
dependence on any third party, pseudonymity, simplicity, familiarity and cross-device compatibility.
However, they are highly susceptible to human error: use of weak passwords, reuse of the same
password across different services, phishing attacs and leaks of unhashed or weakly hashed password
databases are unfortunately commonplace.

The problem is not ignorance of the risks, but rather the inconvenience of the alternatives.
[TODO cite something about the number of passwords that people are able to memorize.]
Password manager products such as 1Password or LastPass make it feasible for users to maintain a
strong, unique password for each service they use. However, even when a unique password is used, an
attacker who succeeds in stealing a password (e.g. by phishing, keylogging, man-in-the-middle attack
or eavesdropping when it is accidentally sent over an unencrypted connection) has access to that
user account until the password is changed, which is practically forever [citation needed].

Password managers also have grave exposure to human error. For example, consider a user who wishes
to log in to an `unimportant' website on an untrusted computer. They may be carrying their encrypted
password database with them on a trusted device (such as their smartphone), in which case their best
option is to enter the encryption passphrase on their smartphone's tiny keyboard, find the record
for the desired website, read the random password from the smartphone screen and type it letter by
letter into the browser on the untrusted computer. By contrast, it seems much more convenient to
\emph{(``just this once'')} type the master passphrase into the untrusted computer to decrypt the
password database there, and copy\&paste the password for the `unimportant' website. Of course this
has the effect of exposing the user's entire password database to an attacker. This characteristic
of password managers -- making it so tempting to commit a grave security mistake -- should be a
cause for concern in an authentication system with mainstream adoption.

\subsection{One-time passwords}

A better solution for users who want to log in to their accounts on untrusted or partially trusted
devices is to use one-time passwords (OTPs). The user must in advance, on a trusted device, download
a list of one-time passwords, or register a cryptographic device that generates a sequence of OTPs,
or register a phone number to which OTPs can be sent on demand. (Systems involving a device or phone
are also known as two-factor authentication.) OTPs are supported by an increasing number of consumer
web services, such as the email providers Gmail, Hotmail and Fastmail, demonstrating consumer demand
for temporary access to their accounts on untrusted devices.

With OTPs, the exposure to attacks is more closely aligned with non-technical users' mental model of
website security [citation needed]: from the time they enter the OTP on an untrusted device until
the time they click the logout button, they are vulnerable to an attacker impersonating them on that
particular website. However, unless the attacker has a means of extending their privileges (e.g. by
granting themselves OAuth access to the account, or exploiting another vulnerability), the exposure
ends when the user's session is invalidated, and the exposure is limited to that particular website.
This is still far from ideal, but it is better than an indefinitely long exposure window, or
exposing the user's accounts on other websites, as is often the case with regular passwords.

The downside of OTPs is that they tend to be so inconvenient to use that very few websites are
willing to adopt them as their only or primary authentication mechanism. Especially when logging in
on a trusted device, it is widely considered too much effort for the user to pull a smartphone or a
piece of paper out of their pocket every time they want to log in, and to print off a new list of
passwords before the old one runs out. Thus OTP authentication tends to exist only alongside other
authentication mechanisms that are more convenient, and OTPs tend to be used only by a particularly
security-conscious subset of users. [citation needed]

\subsection{OpenID and OAuth}

OpenID~\cite{OpenID} is probably the best-known attempt to relieve users from the burden of
remembering or storing passwords, and as of 2009 was used for authentication on over 9 million
websites~\cite{OpenID09}.  Although in theory OpenID allows the user to choose any identity provider
by entering a URL, in practice many implementations provide a menu with a small selection of
well-known OpenID providers. In cases where the OpenID provider is also an OAuth service
provider~\cite{OAuth}, the line between OpenID and OAuth becomes blurred from the user's point of
view. Some providers, such as Google Federated Login~\cite{GoogleOpenID}, even combine the
protocols.

OpenID does not solve the fundamental problem of authentication: it only delegates the problem to
the OpenID provider, who must then use some other authentication method (most commonly a password,
possibly in conjunction with a code from a hardware token). This means that all the security
characteristics discussed above apply to the OpenID provider, with an additional privilege
escalation problem: if an attacker is able to gain access to the user's OpenID provider account,
they can access any account associated with that OpenID. Sooner or later, a user is likely to be
tempted to type their OpenID provider password into an untrusted device, and the authentication
system would amplify, rather than dampen, this error.

Moreover, in OpenID, the relying party needs to trust the OpenID provider to correctly authenticate
the user. If the OpenID provider is compromised, the relying party has no way of detecting
unauthorized logins from that OpenID provider. If the OpenID provider goes out of business, many
users of that provider will lose their ability to log in (unless they had set up delegation of their
identity URL, which is unrealistic to expect of non-technical users).

Delegated authentication systems like OpenID are thus lacking in some of the desirable properties of
authentication systems outlined above.

\subsection{Client certificates}
\subsection{BrowserID}
